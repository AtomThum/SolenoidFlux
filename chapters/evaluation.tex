\chapter{Evaluation of the magnetic flux}

\section{Magnetic flux of an actual solenoid}

\subsection{Evaluating the flux on the solenoid's contour}

The magnetic dipole is hovering at height $h$ above the middle of the solenoid. The $\vv{A}$ field of the perfect magnetic dipole is given by
\begin{equation}
	\vv{A}(\brc) = \frac{\bperm}{4\cpi}\frac{\vv{m}\times\brc}{\rc^3}. \label{eq:A-field-magnetic-dipole}
\end{equation}
We denote the vector that points from the origin to the contour as $\vv{l}$. In the solenoid's contour ($\gamma_1$),
\begin{equation}
	\vv{l} = R\sin(\omega\alpha)\xhat + R\cos(\omega\alpha)\yhat + \frac{H}{2\cpi}\alpha\zhat.
\end{equation}
Trivially,
\begin{equation}
	\odv{\vv{l}}{\alpha} = R\omega\cos(\omega\alpha)\xhat - R\omega\sin(\omega\alpha)\yhat + \frac{H}{2\cpi}\zhat. \label{eq:differential-element}
\end{equation}
Geometrically,
\begin{align}
	h\zhat + \brc & = \vv{l}                                                                                     \\
	\brc          & = \vv{l} - h\zhat                                                                            \\
	              & = R\sin(\omega\alpha)\xhat + R\cos(\omega\alpha)\yhat + \ab(\frac{H}{2\cpi}\alpha - h)\zhat,
\end{align}
and
\begin{align}
	\rc & = \sqrt{\ab(R\sin(\omega\alpha))^2 + \ab(R\cos(\omega\alpha))^2 + \ab(\frac{H}{2\cpi}\alpha - h)^2} \\
	    & = \sqrt{R^2 + \ab(\frac{H}{2\cpi}\alpha - h)^2}.
\end{align}
From \cref{eq:A-field-magnetic-dipole},
\begin{align}
	\vv{m} \times \brc & = -m\zhat \times \ab(R\sin(\omega\alpha)\xhat + R\cos(\omega\alpha)\yhat + \ab(\frac{H}{2\cpi}\alpha - h)\zhat) \\
	                   & = \begin{vmatrix}
		                       \xhat               & \yhat               & \zhat                     \\
		                       0                   & 0                   & -m                        \\
		                       R\sin(\omega\alpha) & R\cos(\omega\alpha) & \frac{H}{2\cpi}\alpha - h
	                       \end{vmatrix}                                         \\
	                   & = m\begin{vmatrix}
		                        \xhat               & \yhat               \\
		                        R\sin(\omega\alpha) & R\cos(\omega\alpha)
	                        \end{vmatrix}                                                                    \\
	                   & = mR\ab(\cos(\omega\alpha)\xhat - \sin(\omega\alpha)\yhat).
\end{align}
Therefore,
\begin{equation}
	\vv{A}(\alpha) = \frac{\bperm}{4\cpi}\frac{mR\ab(\cos(\omega\alpha)\xhat - \sin(\omega\alpha)\yhat)}{\ab(R^2 + \ab(\frac{H}{2\cpi}\alpha - h)^2)^\frac{3}{2}}
\end{equation}
And thus, from \cref{eq:differential-element}
\begin{align}
	\vv{A}\vdot\odv{\vv{l}}{\alpha} & = \begin{multlined}[t]
		                                    \frac{\bperm mR}{4\cpi}\ab(R^2 + \ab(\frac{H}{2\cpi}\alpha - h)^2)^{-\frac{3}{2}} \\
		                                    \ab(\cos(\omega\alpha)\xhat - \sin(\omega\alpha)\yhat) \vdot \ab(R\omega\cos(\omega\alpha)\xhat - R\omega\sin(\omega\alpha)\yhat + \frac{H}{2\cpi}\zhat)
	                                    \end{multlined} \nonumber \\
	                                & = \frac{\bperm mR}{4\cpi}\ab(R^2 + \ab(\frac{H}{2\cpi}\alpha - h)^2)^{-\frac{3}{2}}R\omega\ab(\cos^2(\omega\alpha) + \sin^2(\omega\alpha))                                       \\
	                                & = \frac{\bperm mR^2\omega}{4\cpi}\ab(R^2 + \ab(\frac{H}{2\cpi}\alpha - h)^2)^{-\frac{3}{2}}
\end{align}
Thus, the magnetic flux that's generated by the contour $\gamma_1$ is given by
\begin{align}
	\int_{\gamma_1}\vv{A}\vdot\odif{\vv{l}} & = \int_{0}^{2\cpi}\vv{A}\vdot\odv{\vv{l}}{\alpha}\odif{\alpha}                                                           \\
	                                        & = \int_{0}^{2\cpi}\frac{\bperm mR^2\omega}{4\cpi}\ab(R^2 + \ab(\frac{H}{2\cpi}\alpha - h)^2)^{-\frac{3}{2}}\odif{\alpha} \\
	                                        & = \frac{\mu_0 mR^2\omega}{4\cpi}\int_{0}^{2\cpi}\ab(R^2 + \ab(\frac{H}{2\cpi}\alpha - h)^2)^{-\frac{3}{2}}\odif{\alpha}
\end{align}
This integral can be easily solved by a simple substitution. Let $u = \frac{H}{2\cpi}\alpha - h$, then $\odif{u} = \frac{H}{2\cpi}\odif{\alpha}$. The bounds are changed from $\alpha = 0 \appr u = -h$, and $\alpha = 2\cpi \appr u = H - h$.
\begin{align}
	 & \frac{\mu_0 mR^2\omega}{4\cpi}\int_{0}^{2\cpi}\ab(R^2 + \ab(\frac{H}{2\cpi}\alpha - h)^2)^{-\frac{3}{2}}\odif{\alpha} \nonumber \\
	 & = \frac{\mu_0 mR^2\omega}{4\cpi}\vdot\frac{2\cpi}{H}\int_{-h}^{H - h}\ab(R^2 + u^2)^{-\frac{3}{2}}\odif{u}                      \\
	 & = \frac{\mu_0 mR^2\omega}{4\cpi}\vdot\peval{\frac{u}{R^2(R^2 + u^2)^{\frac{1}{2}}}}_{u = -h}^{u = H - h}                        \\
	 & = \frac{\mu_0 mR^2\omega}{2H}\ab(\frac{H - h}{\sqrt{R^2 + (H - h)^2}} + \frac{h}{\sqrt{R^2 + h^2}})
\end{align}

\subsection{Evaluating the flux on the solenoid's cable}

The contour $\gamma_2$ to $\gamma_4$ does not contribute to the magnetic flux. This is because the vector $\vv{m} \times \brc$ and $\vv{l}$ are orthogonal. The first one lies on the $x$ axis, but the other lies along the $y$ axis only. Therefore, the only part that contributes to the magnetic flux is $\gamma_1$:
\begin{equation}
	\mflux = \frac{\mu_0 mR^2\omega}{2H}\ab(\frac{H - h}{\sqrt{R^2 + (H - h)^2}} + \frac{h}{\sqrt{R^2 + h^2}})
\end{equation}

\section{Magnetic flux of a stacked circular wire}

Each of the loop contributes a magnetic flux $\mflux(h)$ where $h$ is the height from the center of the wire to the magnet. The contour of a circular wire at radius $R$ that's at $z = 0$ is parameterized by
\begin{equation}
	\gamma: (R\cos(\alpha), R\sin(\alpha), 0) \quad \alpha \in [0, 2\cpi]
\end{equation}
Therefore,
\begin{equation}
	\vv{l} = R\cos(\alpha)\xhat + R\sin(\alpha)\yhat,
\end{equation}
and thus,
\begin{equation}
	\odv{\vv{l}}{\alpha} = -R\sin(\alpha)\xhat + R\cos(\alpha)\yhat.
\end{equation}
Geometrically,
\begin{align}
	h\zhat + \brc & = \vv{l}                                             \\
	\brc          & = -R\sin(\alpha)\xhat + R\cos(\alpha)\yhat - h\zhat,
\end{align}
and
\begin{equation}
	\rc = \sqrt{R^2\sin^2(\alpha) + R^2\cos(\alpha) + h^2} = \sqrt{R^2 + h^2}.
\end{equation}
From \cref{eq:A-field-magnetic-dipole},
\begin{align}
	\vv{m} \times \brc & = \begin{vmatrix}
		                       \xhat         & \yhat         & \zhat \\
		                       0             & 0             & -m    \\
		                       R\cos(\alpha) & R\sin(\alpha) & -h
	                       \end{vmatrix}         \\
	                   & = m\begin{vmatrix}
		                        \xhat         & \yhat         \\
		                        R\cos(\alpha) & R\sin(\alpha)
	                        \end{vmatrix}                \\
	                   & = mR\ab(\sin(\alpha)\xhat - \cos(\alpha)\yhat);
\end{align}
therefore,
\begin{align}
	\vv{A}(\alpha) = \frac{\bperm mR}{4\cpi} \frac{\sin(\alpha)\xhat - \cos(\alpha)\yhat}{(R^2 + h^2)^{\frac{3}{2}}},
\end{align}
and
\begin{align}
	\vv{A}(\alpha)\vdot\odv{\vv{l}}{\alpha} & = \begin{multlined}[t]
		                                            \frac{\bperm mR}{4\cpi}(R^2 + h^2)^{-\frac{3}{2}} \\
		                                            \vdot \ab(\sin(\alpha)\xhat - \cos(\alpha)\yhat) \vdot \ab(-R\sin(\alpha)\xhat + R\cos(\alpha)\yhat)
	                                            \end{multlined} \\
	                                        & = \frac{\bperm mR^2}{4\cpi}(R^2 + h^2)^{-\frac{3}{2}} \ab(\sin^2(\alpha) + \cos^2(\alpha))                                                        \\
	                                        & = \frac{\bperm mR^2}{4\cpi}\frac{1}{(R^2 + h^2)^{\frac{3}{2}}}.
\end{align}
The magnetic flux of a single loop is then
\begin{align}
	\int\vv{A}(\alpha)\vdot\odv{\vv{l}}{\alpha}\odif{\alpha} & = \int_{0}^{2\cpi}\frac{\bperm mR^2}{4\cpi}\frac{1}{(R^2 + h^2)^{\frac{3}{2}}}\odif{\alpha} \\
	                                                         & = \frac{\bperm mR^2}{4\cpi}\frac{1}{(R^2 + h^2)^{\frac{3}{2}}}\int_{0}^{2\cpi}\odif{\alpha} \\
	                                                         & = \frac{\bperm mR^2}{4\cpi}(R^2 + h^2)^{-\frac{3}{2}}2\cpi                                  \\
	                                                         & = \frac{\bperm mR^2}{2}(R^2 + h^2)^{-\frac{3}{2}}.
\end{align}

In \cref{fig:stacked-circles}, the top-most loop is at $z = H$. Therefore, the distance between the magnet to each loop from the bottom-most to top-most is
\begin{equation}
	h, h - \frac{H}{\omega}, h - \frac{2H}{\omega}, \dots, h - \frac{(\omega - 1)H}{\omega}, h - H.
\end{equation}
Therefore, the total magnetic flux given by the stacked circle approximation is
\begin{align}
	\frac{\bperm mR^2}{2}\sum_{n = 0}^{\omega}\ab(h - \frac{nH}{\omega})^{-\frac{3}{2}}.
\end{align}
